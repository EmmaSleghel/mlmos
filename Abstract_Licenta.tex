\documentclass{article}
\usepackage[T1]{fontenc}
\usepackage[utf8]{inputenc}

\usepackage[romanian]{babel}
\usepackage{combelow}
\usepackage{newunicodechar}



\begin{document}

\title{
Analiza sentimentelor pe text\\
{\normalsize
Clasificarea comentariilor după gradul de toxicitate
}
}

\author{
\IEEEauthorblockN{Autor Emma Șleghel} \\
\IEEEauthorblockA{Facultatea de Informatică,\\
Universitatea Alexandru Ioan Cuza, \\
Iași
}
}

\maketitle

\begin{abstract}

{\large
În ultimii ani, din cauza creșterii masive a rețelelor de socializare, comunicarea a suferit schimbări majore. Aceste schimbări, asemenea multora, prezintă atât avantaje, cât și dezavantaje. Unul dintre dezavantajele majore care au apărut în urma acestui progres tehnologic este acela că s-a pierdut controlul asupra modului în care utilizatorii se adresează unul altuia, întrucât aceștia pot publica cu ușurință, și, în general, fără consecințe directe asupra propriei persoane, opinii sau comentarii negative, jignitoare sau chiar toxice. 
Acest lucru este cu atât mai îngrijorător, cu cât conștientizăm faptul că un mare procent din ceea ce circulă pe aceste rețele de socializare este public, permițându-se astfel influențarea unei mase mari de persoane prin postări cu un grad de veridicitate discutabil. \\
\tab Prin urmare, scopul acestui proiect este construirea unui model capabil să evalueze polaritatea (pozitivă, negativă, sau neutră) a comentariilor de pe rețelele de socializare, dar și să determine gradul de toxicitate al acestora. Pentru a realiza acest lucru, vor fi testați mai mulți algoritmi de Data Mining și Machine Learning, urmând ca în final să fie selectați cei mai eficienți pentru clasificarea unui set de date de dimensiuni mari. \\
Prima secțiune reprezintă o descriere mai detaliată a subiectului, pentru oferirea unui context mai bun. Secțiunea a doua prezintă obiectivele proiectului, urmată de modul de implementare detaliat în cea de-a treia secțiune. Secțiunea IV se referă la rezultatele obținute în urma implementării proiectului, iar concluziile finale sunt prezentate în secțiunea VI.
}

\end{abstract}


\end{document}


